\documentclass{article}

% Language setting
% Replace `english' with e.g. `spanish' to change the document language
\usepackage[spanish]{babel}

% Set page size and margins
% Replace `letterpaper' with `a4paper' for UK/EU standard size
\usepackage[a4paper,top=2cm,bottom=2cm,left=2cm,right=2cm,marginparwidth=1.75cm]{geometry}

% Useful packages
\usepackage{amsmath}
\usepackage{graphicx}
\usepackage[colorlinks=true, allcolors=blue]{hyperref}
\usepackage{tabularx} % En el preámbulo
\usepackage{booktabs} % Opcional: mejora el diseño de las líneas
\usepackage{caption}  % Mejora el formato de las tablas

\title{Aplicación web con Stable Diffusion}
\author{Jose Luis Campo Rivas}
\date{\vspace{-2ex}}

\begin{document}
\maketitle

\section{Plan de Desarrollo}

\subsection{Resumen}

\subsubsection*{Descripción}
Este proyecto consiste en una aplicación web que permite a los usuarios generar imágenes a partir de descripciones de texto (prompts) utilizando modelos de inteligencia artificial. La generación se basa en el modelo Stable Diffusion 2.1 \cite{rombach2022high, stablediffusion21}, un modelo de difusión entrenado para crear imágenes realistas y de alta calidad a partir de lenguaje natural. La arquitectura del sistema se divide en dos partes principales: un backend basado en FastAPI \cite{ramirez2023fastapi} y un frontend simple en HTML y JavaScript.

\subsubsection*{Objetivos}

\begin{itemize}
  \item Proporcionar una interfaz web accesible para generar imágenes a partir de texto.
  \item Implementar una API REST eficiente que aproveche modelos preentrenados de generación de imágenes.
  \item Facilitar la despliegue y portabilidad de la aplicación mediante contenedores Docker \cite{docker2023}.
  \item Aprovechar aceleración por GPU cuando esté disponible, para mejorar el rendimiento de la generación.
\end{itemize}

\subsubsection*{Problemática abordada}
Con el auge de la inteligencia artificial generativa, muchas personas y desarrolladores buscan herramientas para crear imágenes desde texto de forma accesible. Sin embargo:

\begin{itemize}
  \item Muchos modelos son complejos de ejecutar localmente debido a sus requisitos técnicos (GPU, librerías especializadas).

  \item Las interfaces existentes suelen ser demasiado técnicas o dependen de plataformas externas.

  \item Se necesita una solución autoalojable, ligera y sencilla, que permita experimentar con generación de imágenes sin depender de terceros.
\end{itemize}

Este proyecto responde a esa necesidad, ofreciendo una solución modular, fácil de desplegar y de usar tanto para usuarios como para desarrolladores.

\subsection{Estado del arte}

\subsubsection*{Tecnologías utilizadas}

Este proyecto se enmarca en el contexto de la computación distribuida y el aprendizaje profundo, aprovechando herramientas modernas para el despliegue, inferencia y consumo de modelos de generación de imágenes. A continuación se describen las principales tecnologías involucradas:
\begin{itemize}
  \item \textbf{Stable Diffusion:} modelo de difusión de código abierto diseñado para la generación de imágenes a partir de texto. Es altamente eficiente en términos de calidad y tamaño, y se encuentra entre los modelos más utilizados en el ámbito del arte generativo \cite{rombach2022high, stablediffusion21, vonPlaten2023diffusers}.

  \item \textbf{FastAPI:} framework web moderno para construir APIs en Python, basado en ASGI. Destaca por su rendimiento, su compatibilidad con Python async/await y su integración con herramientas de documentación como Swagger/OpenAPI \cite{ramirez2023fastapi, openapi2021}.

  \item \textbf{Docker:} plataforma de contenedores que permite encapsular aplicaciones y sus dependencias, facilitando su ejecución en distintos entornos. Es ampliamente adoptada en despliegues de aplicaciones de IA, por su facilidad de uso y portabilidad \cite{docker2023}.

  \item \textbf{Hugging Face Diffusers:} biblioteca que facilita el uso de modelos generativos como Stable Diffusion, ofreciendo interfaces estándar y compatibilidad con aceleración por GPU \cite{vonPlaten2023diffusers}.
\end{itemize}

\subsubsection*{Comparativa de herramientas de orquestación}

Para el despliegue y escalado de este tipo de aplicaciones, existen varias tecnologías que permiten gestionar entornos distribuidos, contenedores y recursos de hardware (como GPUs). A continuación, se presenta una comparativa entre las más relevantes:

\begin{table}[!ht]
    \centering
    \renewcommand{\arraystretch}{1.2} % Espaciado entre filas
    \begin{tabularx}{\textwidth}{|l|X|X|X|}
        \hline
        \textbf{Herramienta} & \textbf{Enfoque principal} & \textbf{Ventajas destacadas} & \textbf{Limitaciones} \\ \hline
        \textbf{Docker} & Contenerización individual & Simple, portable, ideal para desarrollo local & No gestiona clústeres ni escalado automático \\ \hline
        \textbf{Kubernetes} & Orquestación de contenedores & Escalado automático, alta disponibilidad, gestión avanzada de recursos & Complejidad inicial, curva de aprendizaje elevada \cite{kubernetes2023} \\ \hline
        \textbf{OpenShift} & Plataforma de aplicaciones sobre Kubernetes & Seguridad integrada, CI/CD, interfaz amigable & Requiere infraestructura más compleja y recursos avanzados \cite{openshift2023} \\ \hline
    \end{tabularx}
    \caption{Comparativa de herramientas de contenedores}
\end{table}

Para un entorno de desarrollo o despliegue sencillo (como en este proyecto), Docker ofrece una solución liviana y efectiva. No obstante, en contextos empresariales o de alta demanda (Big Data, alta concurrencia, multiclúster), Kubernetes o OpenShift son más adecuados por sus capacidades de orquestación, balanceo de carga y gestión distribuida.

\subsubsection*{Relevancia en el contexto de computación distribuida, aprendizaje profundo y Big Data}

La combinación de estas herramientas responde a los retos actuales del desarrollo de aplicaciones de inteligencia artificial, donde la escalabilidad, la portabilidad y el rendimiento son claves. En particular:

\begin{itemize}
  \item \textbf{Aprendizaje profundo} requiere hardware especializado (como GPUs) y entornos reproducibles para ejecutar modelos complejos como Stable Diffusion.

  \item \textbf{Big Data} se relaciona con la necesidad de manejar grandes volúmenes de imágenes, logs y modelos. Herramientas como Kubernetes permiten escalar horizontalmente para procesar múltiples solicitudes concurrentes.

  \item \textbf{Computación distribuida} cobra relevancia cuando se necesita paralelizar tareas de inferencia, despliegue o entrenamiento, especialmente en contextos multiusuario o en la nube.
\end{itemize}

Este proyecto se sitúa como una implementación ligera, pero perfectamente integrable dentro de una arquitectura más compleja si se combinara con orquestadores como Kubernetes o servicios en la nube.

\subsection{Diagrama de arquitectura}
\subsection{Documento de operaciones}


\section{Implementación del Pipeline CI/CD}


\section{Uso de Modelos Preentrenados}

\newpage
\bibliographystyle{plain}
\bibliography{sample}

\end{document}